% ##################################################
% Unterstuetzung fuer die deutsche Sprache
% ##################################################
%\usepackage{ngerman}
\usepackage[ngerman]{babel}

% ##################################################
% Dokumentvariablen
% ##################################################

% Persoenliche Daten
\newcommand{\docNachname}{Zipperle}
\newcommand{\docVorname}{Michael}
\newcommand{\docStrasse}{Unter Stad 30}
\newcommand{\docOrt}{Bahlingen}
\newcommand{\docPlz}{79353}
\newcommand{\docEmail}{Michael@Zipperle.de}
\newcommand{\docMatrikelnummer}{259564}

% Dokumentdaten
\newcommand{\docTitle}{Untersuchung der \acs{AWS} Mobile Hub}
%\newcommand{\docUntertitle}{} % Kein Untertitel
\newcommand{\docUntertitle}{}
% Arten der Arbeit: Bachelorthesis, Masterthesis, Seminararbeit, Diplomarbeit
\newcommand{\docArtDerArbeit}{Seminararbeit}
%Studiengaenge: Allgemeine Informatik Bachelor, Computer Networking Bachelor,
% Software-Produktmanagement Bachelor, Advanced Computer Scinece Master
\newcommand{\docStudiengang}{Mobile Systeme}
\newcommand{\docAbgabedatum}{06.07.2018}
\newcommand{\docErsterReferent}{Prof. Dr. Christoph Reich}
%\newcommand{\docZweiterReferent}{-} % Wenn es nur einen Betreuer gibt
%\newcommand{\docZweiterReferent}{Norbert Falkenberg (SICK AG)}

% ##################################################
% Allgemeine Pakete
% ##################################################

% Abbildungen einbinden
\usepackage{graphicx}

% Mehrspaltige Listen
\usepackage{multicol}

% Zusaetsliche Sonderzeichen
\usepackage{dingbat}

% Einbindung von PDFs
\usepackage{pdfpages}

% Formatierung von Zahlen
\usepackage{siunitx}

% Farben
\usepackage{color}

%\usepackage[usenames,dvipsnames,svgnames,table]{xcolor}

% Maskierung von URLs und Dateipfaden
\usepackage[hyphens]{url}

% Deutsche Anfuehrungszeichen
\usepackage[babel, german=quotes]{csquotes}

% Pakte zur Index-Erstellung (Schlagwortverzeichnis)
\usepackage{index}
\makeindex

% Ipsum Lorem
% Paket wird nur für das Beispiel gebraucht und kann gelöscht werden
\usepackage{lipsum}

% ##################################################
% Seitenformatierung
% ##################################################
\usepackage[
	portrait,
	bindingoffset=1.5cm,
	inner=2.5cm,
	outer=2.5cm,
	top=3cm,
	bottom=2cm,
	%includeheadfoot
	]{geometry}

% ##################################################
% Kopf- und Fusszeile
% ##################################################

\usepackage{fancyhdr}

\pagestyle{fancy}
\fancyhf{}
\fancyhead[EL,OR]{\sffamily\thepage}
\fancyhead[ER,OL]{\sffamily\leftmark}

\fancypagestyle{headings}{}

\fancypagestyle{plain}{}

\fancypagestyle{empty}{
  \fancyhf{}
  \renewcommand{\headrulewidth}{0pt}
}

%Kein "Kapitel # NAME" in der Kopfzeile
\renewcommand{\chaptermark}[1]{
	\markboth{#1}{}
   	\markboth{\thechapter.\ #1}{}
}

% ##################################################
% Schriften
% ##################################################

% Stdandardschrift festlegen
\renewcommand{\familydefault}{\sfdefault}

% Standard Zeilenabstand: 1,5 zeilig
\usepackage{setspace}
\onehalfspacing 

% Schriftgroessen festlegen
\addtokomafont{chapter}{\sffamily\large\bfseries} 
\addtokomafont{section}{\sffamily\normalsize\bfseries} 
\addtokomafont{subsection}{\sffamily\normalsize\mdseries} 
\addtokomafont{subsubsection}{\sffamily\normalsize\mdseries} 
\addtokomafont{caption}{\sffamily\normalsize\mdseries} 

% ##################################################
% Inhaltsverzeichnis / Allgemeine Verzeichniseinstellungen
% ##################################################

\usepackage{tocloft}

% Punkte auch bei Kapiteln
\renewcommand{\cftchapdotsep}{3}
\renewcommand{\cftdotsep}{3}

% Schriftart und -groesse im Inhaltsverzeichnis anpassen
\renewcommand{\cftchapfont}{\sffamily\normalsize}
\renewcommand{\cftsecfont}{\sffamily\normalsize}
\renewcommand{\cftsubsecfont}{\sffamily\normalsize}
\renewcommand{\cftchappagefont}{\sffamily\normalsize}
\renewcommand{\cftsecpagefont}{\sffamily\normalsize}
\renewcommand{\cftsubsecpagefont}{\sffamily\normalsize}

%Zeilenabstand in den Verzeichnissen einstellen
\setlength{\cftparskip}{.5\baselineskip}
\setlength{\cftbeforechapskip}{.1\baselineskip}

% ##################################################
% Abbildungsverzeichnis und Abbildungen
% ##################################################

\usepackage{caption}
\captionsetup{
	format=hang,
	justification=centering,
	singlelinecheck=off
}
\usepackage{wrapfig}

% Nummerierung von Abbildungen
\renewcommand{\thefigure}{\arabic{figure}}
\usepackage{chngcntr}
\counterwithout{figure}{chapter}

% Abbildungsverzeichnis anpassen
\renewcommand{\cftfigpresnum}{Abbildung }
\renewcommand{\cftfigaftersnum}{:}

% Breite des Nummerierungsbereiches [Abbildung 1:]
\newlength{\figureLength}
\settowidth{\figureLength}{\bfseries\cftfigpresnum\cftfigaftersnum}
\setlength{\cftfignumwidth}{\figureLength}
\setlength{\cftfigindent}{0cm}

% Schriftart anpassen
\renewcommand\cftfigfont{\sffamily}
\renewcommand\cftfigpagefont{\sffamily}

% Bilder nebeneinander positionieren
\usepackage{subfig}

% Verhindern des Einrückens nach einem Absatz
\setlength{\parindent}{0em} 

% ##################################################
% Tabellenverzeichnis und Tabellen
% ##################################################

% Nummerierung von Tabellen
\renewcommand{\thetable}{\arabic{table}}
\counterwithout{table}{chapter}

% Tabellenverzeichnis anpassen
\renewcommand{\cfttabpresnum}{Tabelle }
\renewcommand{\cfttabaftersnum}{:}

% Breite des Nummerierungsbereiches [Abbildung 1:]
\newlength{\tableLength}
\settowidth{\tableLength}{\bfseries\cfttabpresnum\cfttabaftersnum}
\setlength{\cfttabnumwidth}{\tableLength}
\setlength{\cfttabindent}{0cm}

%Schriftart anpassen
\renewcommand\cfttabfont{\sffamily}
\renewcommand\cfttabpagefont{\sffamily}

% Unterdrueckung von vertikalen Linien
\usepackage{booktabs}

% Erlauben von mehrspaltigen Kommentaren
\usepackage{comment}

% ##################################################
% Listings (Quellcode)
% ##################################################

\usepackage{listings}
\lstset{
	language=java,
	backgroundcolor=\color{white},
	breaklines=true,
	prebreak={\carriagereturn},
 	breakautoindent=true,
 	numbers=left,
 	numberstyle=\tiny,
 	stepnumber=2,
 	numbersep=5pt,
 	keywordstyle=\color{blue},
   	commentstyle=\color{green},   
   	stringstyle=\color{gray}
}
  	
% ##################################################
% Theoreme
% ##################################################
  	
% Einheiten und Formeln  	
\usepackage{siunitx}

% Umgebung fuer Beispiele
\newtheorem{beispiel}{Beispiel}

% Umgebung fuer These
\newtheorem{these}{These}

% Umgebung fuer Definitionen
\newtheorem{definition}{Definition}
  	
% ##################################################
% Literaturverzeichnis
% ##################################################

%\usepackage{bibgerm}
%\usepackage[style=alphabetic,backend=biber,sorting=ynt,urldate=iso8601,date=iso8601]{biblatex}
\usepackage[style=numeric, sorting=none]{biblatex}
\addbibresource{bibtex.bib}

% ##################################################
% Abkuerzungsverzeichnis
% ##################################################

\usepackage[printonlyused]{acronym}

% ##################################################
% Quellcodeverzeichnis und Quellcode Formatierung
% ##################################################
\usepackage{lstautogobble}
\renewcommand{\lstlistingname}{Codeauszug}% Listing -> Algorithm
\renewcommand{\lstlistlistingname}{Quellcodeverzeichnis}% List of Listings -> List of Algorithms
% JSON
\lstdefinestyle{json}{
	basicstyle=\small,
	string=[s]{"}{"},
	stringstyle=\color{blue},
	comment=[l]{:},
	commentstyle=\color{black},
	autogobble=true,
	frame=tb,
	showspaces=false,
	showstringspaces=false
}

% XML
\definecolor{gray}{rgb}{0.4,0.4,0.4}
\definecolor{darkblue}{rgb}{0.0,0.0,0.6}
\definecolor{cyan}{rgb}{0.0,0.6,0.6}

\lstdefinestyle{xml}
{
	basicstyle=\ttfamily,
	columns=fullflexible,
	showstringspaces=false,
	commentstyle=\color{gray}\upshape,
	morestring=[b]",
	morestring=[s]{>}{<},
	morecomment=[s]{<?}{?>},
	stringstyle=\color{black},
	identifierstyle=\color{darkblue},
	keywordstyle=\color{cyan},
	morekeywords={xmlns,version,type}% list your attributes here
}

% JAVA
\definecolor{pblue}{rgb}{0.13,0.13,1}
\definecolor{pgreen}{rgb}{0,0.5,0}
\definecolor{pred}{rgb}{0.9,0,0}
\definecolor{pgrey}{rgb}{0.46,0.45,0.48}

\lstdefinestyle{java}{
	language=Java,
	showspaces=false,
	showtabs=false,
	breaklines=true,
	showstringspaces=false,
	breakatwhitespace=true,
	commentstyle=\color{pgreen},
	keywordstyle=\color{pblue},
	stringstyle=\color{pred},
	basicstyle=\ttfamily,
	moredelim=[il][\textcolor{pgrey}]{$$},
	moredelim=[is][\textcolor{pgrey}]{\%\%}{\%\%},
	frame=tb,
}

% ##################################################
% PDF / Dokumenteninternelinks
% ##################################################

\usepackage[
	colorlinks=false,
   	linkcolor=black,
   	citecolor=black,
  	filecolor=black,
	urlcolor=black,
    bookmarks=true,
    bookmarksopen=true,
    bookmarksopenlevel=3,
    bookmarksnumbered,
    plainpages=false,
    pdfpagelabels=true,
    hyperfootnotes,
    pdftitle ={\docTitle},
    pdfauthor={\docVorname~\docNachname},
    pdfcreator={\docVorname~\docNachname}]{hyperref}