\documentclass[journal]{IEEEtran}
\usepackage{blindtext}
\let\labelindent\relax
\usepackage[inline]{enumitem}
\usepackage{graphicx}
\usepackage[acronym]{glossaries}
\usepackage{subcaption}
\usepackage[bookmarksopen, bookmarksdepth=2, breaklinks=true]{hyperref}
\let\labelindent\relax

% *** GRAPHICS RELATED PACKAGES ***
%
\ifCLASSINFOpdf
\else
\fi


\newacronym{ble}{BLE}{Bluetooth Low Energy}

\hyphenation{op-tical net-works semi-conduc-tor}


\begin{document}
\title{Workshop: Verbesserung der Mensch-Maschinen-Interaktion durch Emotion Tracking}

\author{\begin{center}
\begin{tabular}{c c} 
 Marius Becherer & Michael Zipperle \\ 
 \textit{259158} & \textit{259564} \\
 Marius.Becherer@hs-furtwangen.de & Michael.Zipperle@hs-furtwangen.de \\
\end{tabular}
\end{center}}%
        
%\author{Christian Laustsen, \textit{20176018},
%        Anders Rikvold, \textit{20176009},
%        and Michael Zipperle, \textit{20176059}}% <-this % stops a space
%\thanks{M. Shell is with the Department
%of Electrical and Computer Engineering, Georgia Institute of Technology, Atlanta,
%GA, 30332 USA e-mail: (see http://www.michaelshell.org/contact.html).}% <-this % stops a space
%\thanks{J. Doe and J. Doe are with Anonymous University.}% <-this % stops a space
%\thanks{Manuscript received April 19, 2005; revised January 11, 2007.}}

% The paper headers
\markboth{Hochschule Furtwangen - Ergonomie, June 2018}%
{Hochschule Furtwangen - Ergonomie, June 2018}

% make the title area
\maketitle


\begin{abstract}
%\boldmath
Bei den meisten Interaktion zwischen Mensch und Maschine werden die Emotionen des Nutzers nicht in Betracht gezogen. Jedoch spielen Emotionen eine wichtige Rolle, den diese Beschreiben wie der Nutzer sich fühlt. Durch das Tracken der Emotion eines Nutzers können die Inhalte der Maschine an die Emotionen des Nutzers angepasst werden. Dies soll dazuführen, dass der Nutzer während der Interaktion mit der Maschine positive Emotionen aufweist. Somit kann durch Emotion Tracking eine Verbesserung der Mensch Maschinen Interaktion herbeigeführt werden. Dieser Artikel erläutert die theoretischen Grundlagen um die Emotionen eines Nutzers zu erkennen und darauf zu reagieren. Des Weiteren werden methodische Mittel beschrieben, wie diese Grundlagen einer Gruppe von Personen im Rahmen eines Workshops vermittelt werden können und wie die Ergebnisse eines durchgeführten Workshops aussehen.
\end{abstract}

% Note that keywords are not normally used for peerreview papers.
%\begin{IEEEkeywords}
%IEEEtran, journal, \LaTeX, paper, template.
%\end{IEEEkeywords}

% For peerreview papers, this IEEEtran command inserts a page break and
% creates the second title. It will be ignored for other modes.
\IEEEpeerreviewmaketitle


% *** START OF SECTIONS ***--------------------------------------------

\section{Einführung}
Immer mehr Cloud Service Anbieter bieten ein \gls{mbaas} an. \gls{mbaas} bietet Entwicklern die Möglichkeit, ihre mobilen Anwendungen mit einem Cloud Backend zu verknüpfen. Dieses Backend erlaubt dem Entwickler, die einfache Konfiguration und das Hinzufügen von Cloud Services zu mobilen Anwendungen. Somit können in wenigen Schritten Cloud Services wie Authentifizierung, Datenspeicher, Benachrichtigung, Analystics und vieles mehr hinzugefügt werden. Die Services können über das vom Anbieter bereitgestellte \gls{sdk} in die mobilen Anwendungen integriert werden. Ein \gls{mbaas} wird von Entwicklern immer mehr verwendet, da der Einsatz Zeit spart für die Konfiguration und Wartung der IT-Infrastruktur einer mobilen Anwendung. Dazu kommen weitere Vorteile wie eine fast ein hundert prozentige Ausfallsicherheit und eine automatische Skalierung der Ressourcen. Aktuell gibt es mehrere Anbieter für ein \gls{mbaas}.


\section{Motivation}
\section{Ziele}
\section{Umsetzung: Methoden}
\subsection{Methoden zum Emotion Tracking}\label{MethodenEmotionTracking}
Es gibt verschiedene Methoden um die Emotion eines Nutzers während dessen Interaktion mit einer Maschine zu tracken. Im folgenden werden beispielhaft verschiedene Methoden erläutert:

\subsubsection{Hautwiderstand und Hauttemperatur}
Das Paper "A Suggestion to Improve User-Friendliness Based
on Monitoring Computer User’s Emotions" beschreibt, wie Emotionen eines Nutzers durch dessen Hauttemperatur und Hautwiderstand bestimmt werden können. Die Autoren nutzen dafür ein Temperatur- und Hautwiderstandssensor, die mit einem Arduino verbunden sind. Die Daten der Sensoren werden in einer SQLite Datenbank gespeichert und auf einer Android App ausgegeben. Die Autoren stellten fest, dass eine Änderung der Hauttemperatur bzw. des Hautwiderstands  auf eine Emotionsänderung des Nutzers zurückzuführen ist \cite{EmotionTrackingGSR}.

\subsubsection{Blick}
Die Autoren des Papers "Improving Human-Computer Interaction
by Gaze Tracking" untersuchten, wie das Tracken des Blickes des Nutzers zur Steuerung von Maschinen verwendet werden kann. Unter anderem konnte festgestellt werden, wo und wie lange der Nutzer ein Objekt auf der Maschine betrachtet. Dabei wurde festgestellt, dass durch dieses Verfahren auch die Emotionen eines Nutzers bestimmt werden können. Beispielsweise verändert sich die Pupillengröße bei einer Emotionsänderung des Nutzers. Dabei nutzen die Autoren die integrierte Webcam in einem Laptop, um den Blick und somit die Emotionen eines Nutzers zu Tracken. Somit wird keine zusätzliche Hardware benötigt, wenn das Endgerät des Nutzers bereits eine Kamera integriert hat \cite{EmotionTrackingGaze}.

\subsubsection{Gesichtsausdruck}
Cloud Service Anbieter wie Amazon, IBM und Microsoft bieten Cognitive Services an, darunter auch ein Service zur Emotionserkennung. Abbildung \ref{fig:microsoftgestenerkennung} zeigt eine Live Demo des Service von Microsoft, dabei wird die Emotion "Überraschung" mit einer Wahrscheinlichkeit von 0,93 erkannt. Bei der Live Demo kann ein Bild hochgeladen oder direkt wie Webcam aufgezeichnet werden. Der Service erkennt dann zuerst die Person bzw. Personen und bestimmt zu jeder Person, mit viel Prozent diese mit einer vorgegeben Emotionen übereinstimmt.

\begin{figure}[!h]
	\centering
	\includegraphics[width=0.9\linewidth]{Pictures/Microsoft_Gestenerkennung}
	\caption[Beispiel: Microsoft Azure Emotionserkennung]{Beispiel: Microsoft Azure Emotionserkennung \cite{MicrosoftAzure}}
	\label{fig:microsoftgestenerkennung}
\end{figure}

\subsubsection{Sprachinformationen}
Die Emotionen eines Menschen spiegeln sich in der Sprache dessen wieder. Ein typisches Beispiel hierfür ist, wenn eine Person einen Vortrag hält und dabei sehr verunsichert und aufgeregt ist, die Person spricht oft schnell und verspricht sich gegebenenfalls. Die Autoren des Papers "Speech emotion recognition approaches in human computer interactiong" untersuchten, wie genau können Emotionen eines Nutzers durch die Sprachinformationen bestimmt werden. Dabei extrahierten diese Muster aus mehreren Sprachsignal und bestimmten für dieses Muster die Emotionen des Sprechers. Diese Informationen wurden dann verwendet, um eine künstliche Intelligenz zu trainieren, um somit für ein beliebiges Sprachsignale die zugehörige Emotion vorherzusagen \cite{SpeechInformation}.

\subsection{Workshop Aufgabe}
Im Workshop wollten wir gemeinsam mit allen Teilnehmern so viele Methoden wie möglich zum Emotion Tracking finden. Dazu wurden die in Kapitel \ref{MethodenEmotionTracking} genannten Methoden den Teilnehmern erstmal vorenthalten. Die Aufgabe wurde folgendermaßen gestellt:

\begin{itemize}
	\item Gruppengröße: 4 Personen
	\item Bearbeitungszeit: 10 Minuten
	\item Arbeitsverfahren: Recherche
	\item Beschreibung: Recherchieren Sie im Internet über Methoden zum Tracken der Emotionen einer Person. Diskutieren Sie die Methode in Ihrer Gruppe und notieren Sie Ihre Ergebnisse in dem folgenden Google Formular.
\end{itemize}

\subsection{Workshop Ergebisse}

Es zeigt sich, dass die geplante Zeit für diese Aufgabe ausreichend war. Die Teilnehmer sammelten in den 10 Minuten 23 Methoden zum Emotionen Tracking. Es wurde ein Google Formular verwendet, da somit die gefunden Methoden übersichtlich und ohne großen Aufwand über den Beamer den Teilnehmer präsentiert werden konnten. Dabei wurden die Methoden einzeln durchgegangen, falls ein Name einer Methode nicht selbsterklärend war, wurde der Verfasser gebeten, die Funktionsweise der Methode kurz zu erklären. Im folgenden eine kurze Auflistung der Ergebnisse, die die Methoden aus Kapitel \ref{MethodenEmotionTracking} ergänzen:

\begin{itemize}
	\item Überwachung von Körperfunktionen
	\begin{itemize}
		\item Herzschlag
		\item Gehirnströme
		\item Atmung
		\item Muskelspannung
	\end{itemize}
	\item Eingabeverhalten auf dem Endgerät	
\end{itemize}


\section{Umsetzung: Andwendungsfälle}\label{Umsetzung_Anwendungsfaelle}
Im letzten Kapitel wurden Methoden aufgezeigt, um die Emotionen eines Nutzer während dessen Interaktion mit einer Maschine zu tracken. In diesem Kapitel gilt es herauszufinden, wie das Wissen über die Emotionen eines Nutzers genutzt werden kann, um die Interaktion zwischen Mensch und Maschine zu verbessern. Dazu sollen Anwendungsfälle aufgezeigt werden, bei denen die Nutzung von Emotion Tracking ein Vorteil aufbringt. Für dieses Kapitel ist keine theoretisches Wissen nötig und es kann somit direkt zu einer interaktiven Aufgabe mit den Teilnehmern des Workshops übergegangen werden.

\subsection{Workshop Aufgabe}
Die Teilnehmer des Workshops wurden in drei Gruppen a vier Personen unterteilt, jeder Gruppe wurde eine der folgenden Emotion Tracking Methode zugewiesen:
\begin{itemize}
	\item Gesichtsausdruck
	\item Sprachinformation
	\item Hauttemperatur
\end{itemize}
Die Gruppen wurden aufgefordert, folgende Aufgabe durchzuführen.
\begin{itemize}
	\item Gruppengröße: 4 Personen
	\item Bearbeitungszeit: 20 Minuten
	\item Arbeitsverfahren: Recherche & Entwicklung
	\item Beschreibung: Die folgenden Aufgaben sind in Bezug zu einer bestimmten Emotion Tracking Methode zu bearbeiten:
	\begin{enumerate}
		\item Recherchieren Sie nach bestehenden Anwendungsfällen, bei denen Emotion Tracking zur Verbesserung der HMI eingesetzt wird.
		\item Überlegen Sie sich Anwendungsfälle, bei denen Emotion Tracking zur Verbesserung der HMI eingesetzt werden könnte.
		\item Welche Vor- und Nachteile der Ihnen zugeteilten Methode kommen auf.
	\end{enumerate}
\end{itemize}

\subsection{Workshop Ergebnisse}\label{Umsetzung_Anwendungsfaelle_Ergebnisse}
Nach Bearbeitung der Aufgabe, wurden jede Gruppe aufgefordert ihre Ergebnisse mit Hilfe eines Plakats den anderen Workshop Teilnehmer zu präsentieren. Im folgenden werden die Ergebnisse der Gruppen genauer erläutert.

\subsubsection{Gesichtsausdruck}
Die Gruppe erarbeitete folgende Anwendungsfälle, bei denen der Gesichtsausdruck zum Emotion Tracking verwendet werden kann:
\begin{itemize}
	\item Schmerzerkennung von Patienten im Krankenhaus
	\item Erkennung von kriminellen Machenschaften eines Menschen
	\item Hilfestellung für Blinde
	\item Lernhilfe für Autisten
\end{itemize}
Folgende Vor- und Nachteile kamen bei der Gruppenarbeit auf:
\begin{itemize}
	\item Vorteile:
	\begin{itemize}
		\item In den meisten Maschinen befindet sich heutzutage eine Webcam, die für die Gesichtsausdruckerkennung genutzt werden kann. Somit ist in diesen Fällen keine zusätzliche Hardware nötig und das Emotion Tracking lässt sich einfach einsetzen.		 
	\end{itemize}
	\item Nachteile:
	\begin{itemize}
		\item Ein Mensch kann seinen Gesichtsausdruck einfach manipulieren, in diesen Fällen, könnte durch Emotion Tracking nicht die richtige Emotion des Menschen festgestellt werden.
	\end{itemize}
\end{itemize}

\subsubsection{Sprachinformation}
Die Gruppe erarbeitete folgende Anwendungsfälle, bei denen Sprachinformationen zum Emotion Tracking verwendet werden können:
\begin{itemize}
	\item Alexa Bestellservice: Durch die Analyse der Sprachinformationen kann festgestellt werden, ob der Nutzer wirklich eine Bestellung aufgeben will. Zum Beispiel kann ein Nutzer als Witz sagen "Alexa, bestelle mir Klopapier", Alexa würde durch die Sprachinformationen erkennen, dass es ein Witz ist und antworten "Du hast doch noch genug Klopapier".
	\item Gemütszustand im Auto: Führt der Fahrer eines Autos ein Gespräch während der Fahrt, könnte dieses Gespräch analysiert werden. Kommen Emotionen wie Frust,Ärger oder Müdigkeit beim Fahrer auf, könnte dieser aufgefordert werden, eine Pause einzulegen.
	\item Notruferkennung: Es gibt Fälle, bei denen Notrufe aufgegeben, die nicht der Wahrheit entsprechen. Durch die Emotion der Person, die den Notruf absetzt, kann festgestellt werden, ob die Angaben des Notrufs der Wahrheit entsprechen. Falls bedenken bei Angaben aufkommen, können diese Angaben hinterfragt werden.
	\item  Verhör: Wie bei der Prüfung des Notrufs, könnten bei einem Verhör die Sprachinformationen genutzt werden, um festzustellen, ob eine Person lügt.
	\item Seelsorge: Soziale Roboter können bei der Seelsorge die Emotionen des Menschen einbeziehen und somit diesen besser fördern.
\end{itemize}
Folgende Vor- und Nachteile kamen bei der Gruppenarbeit auf:
\begin{itemize}
	\item Vorteile:
	\begin{itemize}
		\item Diese Methode lässt sich meist ohne zusätzliche Hardware einsetzten, da die meisten Maschinen bereits ein Mikrofon integriert haben.		 
	\end{itemize}
	\item Nachteile:
	\begin{itemize}
		\item Auch die Sprachinformation lassen sich einfach manipulieren. Eine Mensch kann seine Tonlage, Sprechtempo usw. verändern, wodurch die Genauigkeit der Methode sinkt.
		\item Um die Sprachinformationen einer Person analysieren zu können, muss dessen Aussage aufgenommen werden. Somit erhält man nicht nur Information über die Sprache und somit der Emotion sondern auch über den Inhalt der Aussage. Somit würde dieses Methode eine Person ausspionieren.
	\end{itemize}
\end{itemize}

\subsubsection{Hauttemperatur}
Die Gruppe erarbeitete folgende Anwendungsfälle, bei denen die Hauttemperatur zum Emotion Tracking verwendet werden können:
\begin{itemize}
	\item Schlafanalyse: Durch das messen der Hauttemperatur während des Schlafes einer Person, können die Emotion  und somit ein optimale Schlafzeit bestimmt werden.
	\item Stressanlyse: Es kann festgestellt werden, wenn eine Person besonders viel Stress hat. Demnach könnte das Interface einer Maschine angepasst werden (z.B. beruhigende Farben), um den Stress der Person zu verringern.
\end{itemize}
Folgende Vor- und Nachteile kamen bei der Gruppenarbeit auf:
\begin{itemize}
	\item Vorteile:
	\begin{itemize}
		\item Hauttemperatur Sensoren sind billig.
		\item Messmethode sehr einfach.	 
		\item Die Hauttemperatur kann von einer Person nicht einfach manipuliert werden.
	\end{itemize}
	\item Nachteile:
	\begin{itemize}
		\item Meist ist zusätzliche Hardware nötig, die an der Haut einer Person angebracht werden muss.
		\item Verhalten der Hauttemperatur unter dem Einfluss einer Krankheit (z.B. Fiber) oder der Außentemperatur
	\end{itemize}
\end{itemize}

Wie die Ergebnisse der Gruppenarbeit zeigen, ist es nicht einfach die Emotion einer Person in einem Anwendungsfall einzusetzen. Oft limitieren die Methoden dies, da diese zu ungenau, manipulierbar oder zusätzliche Hardware benötigt. Es zeigte sich auch, dass die erarbeiteten Anwendungsfälle der Gruppenarbeit sich auf bestimmte Gebiete reduzieren lassen. Zum einen gibt es Anwendungsfälle, bei denen durch Emotion Tracking die Wahrheit einer Tätigkeit bestimmt werden kann. Zum anderen können Gesundheitsschädliche Emotionen einer Person reduziert oder sogar vermieden werden. 


\section{Fazit}
\begin{figure}[!h]
	\centering
	\includegraphics[width=0.9\linewidth]{Pictures/Fazit_Grafik}
	\caption[Verbesserung der MMI durch Emotion Tracking]{Verbesserung der MMI durch Emotion Tracking}
	\label{fig:fazitgrafik}
\end{figure}

\section{Diskussion}
Am Ende des Workshops wurden mit den Teilnehmern eine offene Diskussion gestartet. Im folgenden werden die einzelnen Diskussionspunkte aufgelistet und die Ergebnisse beschrieben.

\subsection{Kombination verschiedener Emotionen Tracking Methoden}
In Kapitel \ref{Umsetzung_Anwendungsfaelle} zeigten sich Vor- und Nachteile ausgewählter Methoden zum Emotion Tracking. Die Teilnehmer waren sich einig, dass durch die Kombination verschiedener Methoden (beispielsweise Gesichtsausdruck mit Hauttemperatur) können die Nachteile eliminiert werden. Dies steigert die Genauigkeit und verringert die Manipulierbarkeit.

\subsection{Datenschutz - Privatsphäre}
Hierbei muss sich ein Nutzer die Frage stellen, will ich das die Maschine mit der ich interagiere meine Emotionen weiß? Emotion sind sehr sensible Daten und nach einer kurzen Umfrage, wäre kein Teilnehmer des Workshops damit einverstanden, dass bei deren Interaktion mit einer Maschine ihre Emotionen getrackt werden würden. Ebenso stellt sich die Frage, was macht eine Maschine mit den Emotionen eines Nutzers? - Passt es wirklich die Interaktion für den Nutzer an oder sammelt es auch die Daten und verkauft diese an Dritte weiter? Viel Interaktion mit einer Maschine findet heutzutage über ein mobiles Endgerät statt. Dabei hat das mobile Endgerät meist zu wenig Ressourcen um die Emotion des Nutzers zu bestimmen. Somit wird das mobile Endgerät nur als Eingabegerät genutzt und die Auswertung findet in der Cloud statt, was wiederum eine Gefahr für die Privatsphäre für ein Nutzer darstellt.

\subsection{Umsetzungs-Nutzen Faktor}
Kapitel \ref{Umsetzung_Anwendungsfaelle_Ergebnisse} zeigt, dass es eine große Herausforderung ist, die Emotionen eines Nutzers einzusetzen, um die Interaktion zwischen Mensch und Maschine zu verbessern. Nutzer sind unterschiedlich und reagieren somit unterschiedlich auf Änderungen des Interfaces. Somit stellt sich die Frage, lohnt es sich bei der Entwicklung einer Schnittstelle zwischen Mensch und Maschine die Einbeziehung der Emotionen eines Nutzers? Da das Tracken der Emotionen relativ aufwendig ist, kamen die Teilnehmer des Workshops zu dem Schluss, dass der Einsatz nur in bestimmten Anwendungsfällen sinnvoll ist. Diese Anwendungsfälle beschränken sich größten Teils auf das Gesundheitswesen. Dabei kann Menschen mit Behinderung oder Menschen im Alter, die sich einsam fühlen, ein Assistenzsysteme zur Verfügung gestellt werden, das individuell auf deren Emotionen eingehen kann. Für alltägliche Anwendungen, wie der Besuch von verschiedene Webseiten, ist der Einsatz zu aufwändig.



% *** END OF SECTIONS ***---------------------------------------------

% needed in second column of first page if using \IEEEpubid
%\IEEEpubidadjcol

% An example of a floating figure using the graphicx package.
% Note that \label must occur AFTER (or within) \caption.
% For figures, \caption should occur after the \includegraphics.
% Note that IEEEtran v1.7 and later has special internal code that
% is designed to preserve the operation of \label within \caption
% even when the captionsoff option is in effect. However, because
% of issues like this, it may be the safest practice to put all your
% \label just after \caption rather than within \caption{}.
%
% Reminder: the "draftcls" or "draftclsnofoot", not "draft", class
% option should be used if it is desired that the figures are to be
% displayed while in draft mode.
%
%\begin{figure}[!t]
%\centering
%\includegraphics[width=2.5in]{myfigure}
% where an .eps filename suffix will be assumed under latex, 
% and a .pdf suffix will be assumed for pdflatex; or what has been declared
% via \DeclareGraphicsExtensions.
%\caption{Simulation Results}
%\label{fig_sim}
%\end{figure}

% Note that IEEE typically puts floats only at the top, even when this
% results in a large percentage of a column being occupied by floats.


% An example of a double column floating figure using two subfigures.
% (The subfig.sty package must be loaded for this to work.)
% The subfigure \label commands are set within each subfloat command, the
% \label for the overall figure must come after \caption.
% \hfil must be used as a separator to get equal spacing.
% The subfigure.sty package works much the same way, except \subfigure is
% used instead of \subfloat.
%
%\begin{figure*}[!t]
%\centerline{\subfloat[Case I]\includegraphics[width=2.5in]{subfigcase1}%
%\label{fig_first_case}}
%\hfil
%\subfloat[Case II]{\includegraphics[width=2.5in]{subfigcase2}%
%\label{fig_second_case}}}
%\caption{Simulation results}
%\label{fig_sim}
%\end{figure*}
%
% Note that often IEEE papers with subfigures do not employ subfigure
% captions (using the optional argument to \subfloat), but instead will
% reference/describe all of them (a), (b), etc., within the main caption.


% An example of a floating table. Note that, for IEEE style tables, the 
% \caption command should come BEFORE the table. Table text will default to
% \footnotesize as IEEE normally uses this smaller font for tables.
% The \label must come after \caption as always.
%
%\begin{table}[!t]
%% increase table row spacing, adjust to taste
%\renewcommand{\arraystretch}{1.3}
% if using array.sty, it might be a good idea to tweak the value of
% \extrarowheight as needed to properly center the text within the cells
%\caption{An Example of a Table}
%\label{table_example}
%\centering
%% Some packages, such as MDW tools, offer better commands for making tables
%% than the plain LaTeX2e tabular which is used here.
%\begin{tabular}{|c||c|}
%\hline
%One & Two\\
%\hline
%Three & Four\\
%\hline
%\end{tabular}
%\end{table}


% Note that IEEE does not put floats in the very first column - or typically
% anywhere on the first page for that matter. Also, in-text middle ("here")
% positioning is not used. Most IEEE journals use top floats exclusively.
% Note that, LaTeX2e, unlike IEEE journals, places footnotes above bottom
% floats. This can be corrected via the \fnbelowfloat command of the
% stfloats package.









% if have a single appendix:
%\appendix[Proof of the Zonklar Equations]
% or
%\appendix  % for no appendix heading
% do not use \section anymore after \appendix, only \section*
% is possibly needed

% use appendices with more than one appendix
% then use \section to start each appendix
% you must declare a \section before using any
% \subsection or using \label (\appendices by itself
% starts a section numbered zero.)
%


% use section* for acknowledgement
%\section*{Acknowledgment}


%The authors would like to thank...


% Can use something like this to put references on a page
% by themselves when using endfloat and the captionsoff option.
\ifCLASSOPTIONcaptionsoff
  \newpage
\fi



% trigger a \newpage just before the given reference
% number - used to balance the columns on the last page
% adjust value as needed - may need to be readjusted if
% the document is modified later
%\IEEEtriggeratref{8}
% The "triggered" command can be changed if desired:
%\IEEEtriggercmd{\enlargethispage{-5in}}

% references section

% can use a bibliography generated by BibTeX as a .bbl file
% BibTeX documentation can be easily obtained at:
% http://www.ctan.org/tex-archive/biblio/bibtex/contrib/doc/
% The IEEEtran BibTeX style support page is at:
% http://www.michaelshell.org/tex/ieeetran/bibtex/
%\bibliographystyle{IEEEtran}
% argument is your BibTeX string definitions and bibliography database(s)
%\bibliography{IEEEabrv,../bib/paper}
%
% <OR> manually copy in the resultant .bbl file
% set second argument of \begin to the number of references
% (used to reserve space for the reference number labels box)
\begin{thebibliography}{1}
\bibitem{EmotionTrackingGSR}
Keum Young Sung: A Suggestion to Improve User-Friendliness Based
on Monitoring Computer User’s Emotions (2017)
\bibitem{EmotionTrackingGaze}
Zsolt Jank'o, Levente Hajder: Improving Human-Computer Interaction
by Gaze Tracking (2012)
\bibitem{MicrosoftAzure}
"Microsoft Azure Cognitive Services: Emotionserkennung".
\url{https://azure.microsoft.com/de-de/services/cognitive-services/face/#recognition}. Accessed 13.06.2018.
\bibitem{SpeechInformation}
S. Ramakrishnan, Ibrahiem M.M. El Emary: Speech emotion recognition approaches in human computer interactiong (2011)


\end{thebibliography}

% biography section
% 
% If you have an EPS/PDF photo (graphicx package needed) extra braces are
% needed around the contents of the optional argument to biography to prevent
% the LaTeX parser from getting confused when it sees the complicated
% \includegraphics command within an optional argument. (You could create
% your own custom macro containing the \includegraphics command to make things
% simpler here.)
%\begin{biography}[{\includegraphics[width=1in,height=1.25in,clip,keepaspectratio]{mshell}}]{Michael Shell}

% You can push biographies down or up by placing
% a \vfill before or after them. The appropriate
% use of \vfill depends on what kind of text is
% on the last page and whether or not the columns
% are being equalized.

%\vfill

% Can be used to pull up biographies so that the bottom of the last one
% is flush with the other column.
%\enlargethispage{-5in}



% that's all folks
\end{document}


