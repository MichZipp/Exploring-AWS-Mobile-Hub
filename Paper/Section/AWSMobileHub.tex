\section{AWS Mobile Hub}
Die meisten Menschen denken bei Amazon nur an einen Online Shop, bei dem zahlreiche Produkte gekauft werden können. Doch mit \gls{aws} stellt Amazon eine Cloud-Infrastruktur und Cloud Services bereit, welche aktuell zu einer Haupteinnahmequelle von Amazon zählen. \gls{aws} ist sehr erfolgreich, somit zählt Amazon aktuell zu den Marktführern der Cloud Anbieter \cite{statistacloudmarketshare}. Mit \gls{aws} Mobile Hub stellt Amazon ein \gls{mbaas} zur Verfügung, dass wir im folgenden genauer anschauen. \newline 

Mit \gls{aws} Mobile Hub lassen sich einfach und effizient Cloud Services zu einer mobilen Anwendungen hinzufügen. Im folgenden werden die unterstützten Cloud-Services kurz erläutert.


\subsection{Benachrichtigungen und Nutzungsanylyse}
Mit Amazon Pinpoint bietet \gls{aws} ein Cloud Service, um die Nutzer einer Anwendung via Push-Benachrichtigungen, E-Mail oder SMS mit Informationen zu versorgen. Eine Anwendung veröffentlicht neue Push-Benachrichtigungen an ein Thema und alle Abonnenten dieses Themas werden benachrichtigt. Dabei werden nicht die Endgeräte direkt benachrichtigt, da jedes mobile Betriebssystem seinen eigenen Service hat, um die mobilen Endgeräte zu benachrichtigen. Somit wird der Service des jeweiligen Betriebssystems benachrichtigt und dieser leitet diese Benachrichtigung weiter. Des Weiteren können durch Amazon Pinpoint Nutzungsdaten erfasst und analysiert werden. Dies ermöglicht den Entwicklern das Verhalten der Nutzer besser zu verstehen. Es kann unter anderem  analysiert werden, wie ein Nutzer auf eine Benachrichtigung reagiert oder wie viele Nutzer aktuell eingeloggt sind \cite{AmazonPinpoint}.

\subsection{User Sign-in}
Mit Amazon Cognito lassen sich Registrierung, Anmeldung und Zugriffskontrolle eines Nutzers einfach verwalten. Es können sichere und skalierbare Benutzerverzeichnisse angelegt und soziale Identitätsanbieter wie Facebook und Google oder Unternehmens-Identitätsanbieter wie Microsoft Active Directory zur Anmeldung verwendet werden. Zudem beinhaltet es ein Sicherheitskonzept, das die Multifaktor-Authentifizierung sowie die Verschlüsselung der Daten im Speicher und auf dem Übertragungsweg unterstützt. Ebenso lassen sich mit Amazon Cognito die Zugriffsrechte für andere Cloud-Services im Rahmen der Anwendung festlegen \cite{AmazonCognito}.

\subsection{Cloud Logik}
Amazon Lambda bietet eine Plattform in der Cloud auf der Programme ausgeführt werden können, ohne dass der Nutzer den Server bereitstellen und verwalten muss. Dabei lädt der Entwickler lediglich den Programmcode hoch und Lambda übernimmt den Rest. Amazon Lambda skaliert automatisch die Anwendung je nach Verarbeitungslast und der Anzahl an Zugriffen. Zusätzlich lässt sich Amazon Lambda mit anderen Amazon Services verknüpfen, welche durch ein Event eine Anwendung von Amazon Lambda starten können. Daher ist Amazon Lambda perfekt um Logik einer mobilen Anwendung in die Cloud zu verlangen und somit Ressourcen auf dem mobilen Endgeräte einzusparen. Der Programmcode kann in den Programmiersprachen Javascript, Python, Java und C\# entwickelt werden \cite{AmazonLambda}. Des Weiteren lässt sich mit Hilfe von Amazon \gls{api} Gateway eine \gls{rest} \gls{api} bereitstellen, mit der auf Lambda Funktionen zugegriffen werden kann. Dabei stellt Amazon \gls{api} Gateway die Verwaltung des Datenverkehrs, Autorisierung und Zugriffskontrolle, Überwachung und Verwaltung der API-Version bereit \cite{AmazonAPIGateway}.

\subsection{NoSQL Database}
Amazon DynamoDB ist eine NoSQL Datenbank, die für Anwendungen eine konsistente Antwortzeit im einstelligen Millisekundenbereich anbietet. Zusätzlich passt DynamoDB die Kapazität automatisch anhand der Anwendungsanfragen an. Des Weiteren muss der Nutzer keine Datenbankverwaltungsaufgaben vornehmen. DynamoDB stellt automatisch die Hardware und Software bereit und kümmert sich um den Betrieb eines zuverlässigen, verteilten Datenbank-Clusters oder die Skalierung der Daten über mehrere Instanzen. Dabei unterstützt DynamoDB sowohl das Dokumente- als auch Schlüssel-Wert-Speichermodell. DynamoDB kann mit Amazon Lambda verknüpft werden, sodass bei einer Datenänderung (Hinzufügen, Ändern oder Löschen eines Datensatzes) automatisch eine Lambda Anwendung ausgeführt werden kann. Die Lambda Anwendungen kann dann beispielsweise Nutzer über geänderte Daten in der Datenbank informieren \cite{AmazonDynamoDB}. 

\subsection{Daten Speicher}
Amazon S3 ist ein Objektspeicher, der das Speichern und Abrufen von beliebigen Daten für beliebige Anwendungen ermöglicht. Der Objektspeicher kann für Unternehmensanwendungen, mobile Anwendungen (Apps und Webseiten) sowie der Speicherung der Daten von \acs{iot} Geräten (Sensoren) eingesetzt werden. Dabei wird eine Verfügbarkeit der Daten von nahe zu 100\% und eine automatische Skalierung der Datenmenge gewährleistet. Zudem wird ein umfassendes Sicherheitskonzept eingesetzt, um den neuesten Richtlinien für Sicherheit stand zu halten. Das BigData Konzept wird eingesetzt, um schnellst möglich einen bestimmten Datensatz im Speicher zu finden \cite{AmazonE3}.

\subsection{Konversations-Bot}
Amazon Lex ist ein Konversations-Bot der ermöglicht, einen Chat in einer Anwendung zu erzeugen. Dieser Bot basiert auf einer fortgeschrittenen Deep-Learning-Funktion und erlaubt die Umwandlung von Sprache zu Text, Text zu Sprache und das Sprachverständnis. Dies erlaubt die Erzeugung eines realistischen Gesprächs für eine Anwendung und somit ein gutes Benutzererlebnis. Der Einsatz von Bots bietet zahlreiche Einsatzmöglichkeiten um eine Anwendung zu verbessern. Zum Beispiel werden Bots eingesetzt, um Nutzer auf einer Webseite Hilfe zu gewährleisten, falls diese Fragen haben. Ein anderes Beispiel sind Hotlines von großen Unternehmen. Um Mitarbeiterressourcen zu sparen, werden Bots eingesetzt, um die ersten Informationen des Anrufers auszuwerten und um einen passenden Mitarbeiter bereit zu stellen. Erwähnenswert ist auch, dass Amazon Lex die Grundlage für den Sprachassistenten Alexa ist \cite{AmazonLex}.

\subsection{Hosting und Streaming}
Amazon CloudFront stellt ein \gls{cdn} bereit und erlaubt Daten, Videos, Anwendungen und \gls{api}'s weltweit mit wenig Wartezeit, hoher Sicherheit und hohen Übertragungsraten bereitzustellen \cite{AmazonCloudFront}. 

\subsection{Amazon CloudWatch}
Amazon CloudWatch ist ein Service, der die anderen Cloud Services von \gls{aws} überwacht. CloudWatch kann verschiedene Protokolldateien und Metriken von Cloud Services sammeln und überwachen. Des Weiteren kann der Administrator Alarme festlegen, um zeitnah auf evtl. vorliegende Probleme eines Service zu reagieren. Es können Events definiert werden, die bestimmen, wie das Problem eines ausgelösten Signals automatisch behoben werden kann. Zusätzlich bietet CloudWatch einen Einblick in die Auslastung der Ressourcen, die Anwendungsleistung und die Integrität ihrer Betriebsabläufe \cite{AmazonCloudWatch}.
