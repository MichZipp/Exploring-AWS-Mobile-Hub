\section{Anbieter}
Bevor auf \gls{aws} Mobile Hub genauer eingegangen wird, werden in diesem Kapitel die Konkurrenzanbieter für ein \gls{mbaas} kurz erläutert. Welcher Anbieter sich am besten für eine mobilen Anwendung eignet, hängt einerseits von den Anforderungen der mobilen Anwendung und anderseits von den bisherigen Erfahrungen und Kenntnissen des Entwicklers Teams ab.

\subsection{Apple Cloud Kit}
Das Cloud Kit von Apple wurde 2015 als Teil von iOS 8 veröffentlicht und wird Entwicklern kostenlos zur Verfügung gestellt \cite{techbeacon}. Dies soll die Entwicklung von Anwendungen für Apple's iOS, macOS, web, watchOS and Apple TV vereinfachen und vorantreiben. Das \gls{mbaas} unterstützt ausschließlich Plattformen von Apple und fokussiert sich unter anderem auf die Authentifizierung, Analyse und Speicherung von Daten für eine mobile Anwendung \cite{applecloudkit}.

\subsection{Progress Kinvey}
Progress bietet mit Kinvey ein \gls{mbaas}, das mit allen gängigen mobilen Betriebssystem und Technologien eingesetzt werden kann. Kinvey bietet ein Backend für Authentifizierung, Sicherheit und Compliance, Speicherung von Daten und mehr. Ebenso wirbt es mit einer Verkürzung der Entwicklungszeit um bis zu 75\% beim Einsatz ihres \gls{mbaas}. Für einen einzelnen Entwickler ist die Nutzung vorerst kostenlos, kommen mehrere Entwickler dazu und werden mehr Cloud Ressourcen benötigt, können bis zu \euro{2000} pro Monat anfallen \cite{progresskinvey}.

\subsection{Google Firebase}
Google veröffentlicht Firebase 2016 und ist aktuell einer der meist genutzten Anbieter. Dies liegt einerseits an der großen Anzahl an Services, die sich von Authentifizierung, Speicherung von Daten, Hosting bis hinzu maschinelles Lernen, Cloud Funktionen und einer Test Lab erstrecken. Anderseits an der Unterstützung aller gängigen Plattformen sowie der großen Gemeinschaft von Entwicklern \cite{techbeacon}. Auch Firebase bietet ein kostenlos Paket an, dieses ist jedoch nicht für eine Anzahl an Entwicklern beschränkt, sondern anhand der Cloud Ressourcen. Braucht ein Entwickler Team mehr Ressourcen, so wird nach dem gängigen "Pay As You Go" abgerechnet \cite{googlefirebase}.

\subsection{Kumulos}
Kumulos veröffentlichte bereits 2010 ihr \gls{mbaas} und richtet sich hauptsächlich an Freelancer und Agenturen. Es eignet sich unter anderem für Entwickler, die mehrere mobile Anwendungen verwalten und dafür eine zuverlässige Daten Plattform benötigen \cite{canival}. Kumulos hebt sich mit Features wie Crash Reporting, Automatisches Reporting \& Analytics, App Store Optimierung, Agency Console und Push Benachrichtigungen hervor. Ebenso wirbt es mit der einfachen Konfiguration und den günstigen Preiskonzept und unterstützt alle gängigen mobilen Plattformen und Technologien. Aktuell wird es von mehreren Tausend Entwicklern in mehr als 25 Ländern genutzt \cite{kumulos}.

\subsection{Parse}
Parse ist ein OpenSource Projekt und stellt ein \gls{mbaas} bereit. Es umfasst Service wie Authentifizierung, Speicherung von Daten, Analystics und Push Benachrichtigungen und kann auf jeder beliebige IT-Infrastruktur eingesetzt werden. Alle gängigen mobilen Plattformen und Technologien werden unterstützt. Parse hat den großen Vorteil, dass der Entwickler selbst über die Konfiguration und den Standort der IT-Infrastruktur entscheiden kann und ist auch noch kostenlos \cite{parse}. Doch es gibt auch Anbieter wie back4app, die Parse auf ihrer IT-Infrastruktur bereitstellen und im Rahmen eines Abos Entwicklern zur Verfügung stellen \cite{back4app}. 
