\section{Einführung}
Immer mehr Cloud Service Anbieter bieten ein \gls{mbaas} an. \gls{mbaas} bietet Entwicklern die Möglichkeit, ihre mobilen Anwendungen mit Cloud Backend zu verknüpfen. Dieses Backend erlaubt dem Entwickler, die einfache Konfiguration und das Hinzufügen von Cloud Services zu mobile Anwendungen. Somit können in wenigen Schritten Cloud Services wie Authentifizierung, Datenspeicher, Benachrichtigung, Analystics und vieles mehr zu einer mobilen Anwendungen hinzugefügt werden. Die Services können über das vom Anbieter bereitgestellte \gls{sdk} in die mobile Anwendungen integriert werden. Ein \gls{mbaas} wird von Entwicklern immer mehr eingesetzt, da der Einsatz eine Menge Zeit für die Konfiguration und Wartung von der IT-Infrastruktur einer mobilen Anwendung abnimmt. Dazu kommen weitere Vorteile wie eine fast ein hundert prozentige Ausfallsicherheit und eine automatische Skalierung der Ressourcen, die zum Betreiben der Cloud Services notwendig sind, automatisch skaliert. Aktuell gibt es mehrerer Anbieter für \gls{mbaas}, dieser Artikel soll das \gls{mbaas} \gls{aws} Mobile Hub genauer untersuchen. Dabei wird anhand einer Beispielanwendung gezeigt, welche Möglichkeiten dieses bietet und welche Vor- bzw. Nachteile aufkommen. 

